%-------------------------------Collapse Ratio---------------------------------
\section{Collapse Ratio}

The collapse ratio is a dimensionless number defined as the smallest ratio of the
height of a vertex above its opposing triangle to the longest edge of that opposing
triangle across all vertices of the tetrahedron. Figure~\ref{f:tet-height} shows
how the ratio is computed for a single vertex (vertex $3$). Assuming that edge $0-2$
is the longest edge of triangle $0-1-2$, the collapse ratio for vertex $3$ becomes:
\[
  q_{ex} = \frac{h_3}{\normvec{ L_{02}}}.
\]
In general, take $(i,j,k,\ell)$ to be a permutation of $\{0,1,2,3\}$
(i.e., $(i,j,k,\ell)\in\Sf$) and $\normvec{ L_{ab}}$ to be the length of the edge
connecting vertices $a$ and $b$.
Then the collapse ratio may be written
\[
  q = \min_i\left\{\frac{h_i}{\max\left\{\normvec{ L_{jk}},\normvec{ L_{k\ell}},\normvec{ L_{\ell j}}\right\}}\right\}.
\]

The collapse ratio is intended to identify tetrahedra whose vertices are nearly planar (slivers).
Note that $q$ approaches zero as vertex $3$ in
Figure~\ref{f:tet-height} approaches the plane defined by $0-1-2$.
However, this metric can be misleading when the vertex with the smallest projected height
(say $3$ without loss of generality) is not projected interior to triangle $0-1-2$.
In this case, it is possible for $0-1-2$ to have a small area (which increases $q$) but
for edges $0-3$, $1-3$, and $2-3$ to be very long compared to those of triangle $0-1-2$.
Thus slivers can have arbitrarily high collapse ratios.

\tetmetrictable{collapse ratio}%
{$1$}%                                                Dimension
{$[0.1,DBL\_MAX]$}%                                   Acceptable range
{$(0,DBL\_MAX]$}%                                     Normal range
{$[0,DBL\_MAX]$}%                                     Full range
{$\frac{\sqrt{6}}{3}$}%                               Equilateral tet
{\cite{patran:03}}%                                   Citation
{v\_tet\_collapse\_ratio}%                            Verdict function name


